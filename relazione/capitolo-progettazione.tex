\section{XML e XMLSchema}
Si è scelto di adottare XMLSchema al posto di DTD come linguaggio per descrivere la struttura dei file XML perchè più espressivo, e a fini propedeutici, essendo più complesso. Di seguito verranno spiegate alcune scelte riguardanti la struttura.

\paragraph{Scarso utilizzo degli attributi}

I motivi per cui gli attributi sono poco presenti sono tre:
\begin{itemize}
\item Un elemento è maggiormente estendibile rispetto ad un attributo;
\item Un attributo viene utilizzato per descrivere metadati, mentre i dati in se sono inseriti negli elementi;
\end{itemize}

\paragraph{Formato della data}
In molti file è richiesta una data. Si è scelto di salvare tale data in modo esteso, suddividendo l'elemento data in sottoelementi, cosi facendo è possibile indicare dei range di valore accettabili per giorno, mese ed anno. 
Mese viene scritto in lettere e non come numero, si è quindi applicata una \emph{restriction}, partendo da string si accettano solo le stringhe equivalenti ai mesi.
Non è stato utilizzato il tipo \emph{Data} offerto da XMLSchema per evitare di dover utilizzare il formato americano.
Per il tempo è invece stato utilizzato \emph{Data}, precisamente nel file log.xml.

\paragraph{Elementi opzionali}
Nel file riguardante il menù sono presenti elementi non obbligatori. Nel caso di elementi semplici si è scelto di utilizzare l'attributo nillable="true" per indicare che tale elemento è opzionale. Nel file XML dunque l'elemento dovrà sempre essere presente, e nel caso in cui non lo si voglia definire vi si assegnerà xsi:nil="true".  
Nel caso di elementi complessi si è scelto di utilizzare l'attributo minOccurs="0" che permette di non inerire l'elemento nel file XML. Tale scelta è stata fatta per rendere più leggibili i file XML, che risulterebbero altrimenti appesantiti.

Questi approcci sono necessari in quanto lasciare semplicemente vuoto l'elemento comporterà un errore di validazione, tranne che per gli elementi di tipo stringa.

\paragraph{Modello adottato}
Il modello adottato è \emph{Bambole russe} in quanto non si è ritenuto il riutilizzo del codice di primaria importanza.
Non si è aderito a tale modello solamente nel file menu.xsd. Questo perchè sono presenti quattro elementi (antipasti, primi, secondi, dessert) tutti con la stessa struttura interna, si è quindi dato un nome al tipo e lo si è messo esterno.

%\section{Il nostro menu e JQuery}
%La pagina \emph{il nostro menu} ha come obbiettivo visualizzare il menu proposto dal ristorante. 
%Per rendere più pulita e piacevole la visualizzazione si è messo il contenuto all'interno di vari div (uno per portata) non visibili.
%Cliccando sulla portata verranno visualizzati i relativi piatti e cliccando normalmente verranno nuovamente nascosti.
%Questa tecnica è stata utilizzata nei css per desktop, tablet e smartphone. 
%Per la stampa l'intero menù è stato reso visibile.
%L'accessibilità è garantita da due fattori:
%\begin{itemize}
%\item I div sono nascosti tramite una classe avente il seguente codice, che garantisce la lettura del contenuto da parte dello screen reader
%\begin{lstlisting}
%	position: absolute; 
%	overflow:hidden; 
%	height:0;
%\end{lstlisting}
%\item La classe è applicata ai div tramite JQuery. In tal modo se il browser non supporta javascript, o esso è disabilitato il contenuto resta accessibile.
%\end{itemize}