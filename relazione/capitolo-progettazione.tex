\section{XML e XMLSchema}
Si è scelto di adottare XMLSchema al posto di DTD come linguaggio per descrivere la struttura dei file XML perchè più espressivo, e a fini propedeutici, essendo più complesso. Di seguito verranno spiegate alcune scelte riguardanti la struttura.

\paragraph{Scarso utilizzo degli attributi}

I motivi per cui gli attributi sono poco presenti sono tre:
\begin{itemize}
\item Un elemento è maggiormente estendibile rispetto ad un attributo;
\item Un attributo viene utilizzato per descrivere metadati, mentre i dati in se sono inseriti negli elementi;
\end{itemize}

\paragraph{Formato della data}
In molti file è richiesta una data. Si è scelto di salvare tale data in modo esteso, suddividendo l'elemento data in sottoelementi, cosi facendo è possibile indicare dei range di valore accettabili per giorno, mese ed anno. 
Mese viene scritto in lettere e non come numero, si è quindi applicata una \emph{restriction}, partendo da string si accettano solo le stringhe equivalenti ai mesi.
Non è stato utilizzato il tipo \emph{Data} offerto da XMLSchema per evitare di dover utilizzare il formato americano.
Per il tempo è invece stato utilizzato \emph{Data}, precisamente nel file log.xml.

\paragraph{Elementi opzionali}
Nel file riguardante il menù sono presenti elementi non obbligatori. Nel caso di elementi semplici si è scelto di utilizzare l'attributo nillable="true" per indicare che tale elemento è opzionale. Nel file XML dunque l'elemento dovrà sempre essere presente, e nel caso in cui non lo si voglia definire vi si assegnerà xsi:nil="true".  
Nel caso di elementi complessi si è scelto di utilizzare l'attributo minOccurs="0" che permette di non inerire l'elemento nel file XML. Tale scelta è stata fatta per rendere più leggibili i file XML, che risulterebbero altrimenti appesantiti.

Questi approcci sono necessari in quanto lasciare semplicemente vuoto l'elemento comporterà un errore di validazione, tranne che per gli elementi di tipo stringa.

\paragraph{Modello adottato}
Il modello adottato è \emph{Bambole russe} in quanto non si è ritenuto il riutilizzo del codice di primaria importanza.
Non si è aderito a tale modello solamente nel file menu.xsd. Questo perchè sono presenti quattro elementi (antipasti, primi, secondi, dessert) tutti con la stessa struttura interna, si è quindi dato un nome al tipo e lo si è messo esterno.

%\section{Il nostro menu e JQuery}
%La pagina \emph{il nostro menu} ha come obbiettivo visualizzare il menu proposto dal ristorante.
%Per rendere più pulita e piacevole la visualizzazione si è messo il contenuto all'interno di vari div (uno per portata) non visibili.
%Cliccando sulla portata verranno visualizzati i relativi piatti e cliccando normalmente verranno nuovamente nascosti.
%Questa tecnica è stata utilizzata nei css per desktop, tablet e smartphone. 
%Per la stampa l'intero menù è stato reso visibile.
%L'accessibilità è garantita da due fattori:
%\begin{itemize}
%\item I div sono nascosti tramite una classe avente il seguente codice, che garantisce la lettura del contenuto da parte dello screen reader
%\begin{lstlisting}
%	position: absolute; 
%	overflow:hidden; 
%	height:0;
%\end{lstlisting}
%\item La classe è applicata ai div tramite JQuery. In tal modo se il browser non supporta javascript, o esso è disabilitato il contenuto resta accessibile.
%\end{itemize}
\paragraph{Social}
In ogni pagina, con particolare risalto nella pagina contattaci, è stato reso disponibile un link a Facebook Twitter, Google plus e TripAdvisor. Le motivazioni sono spiegate nella sezione Abstract.
I link non portano a nessuna pagina, in quanto al momento tali pagine non sono presenti, ma verranno create non appena il sito verà accettato e reso pubblico.

\section{Sezione d' amministrazione}
In questa sezione sarà possibile accedere solamente dopo aver effettuato la login nell' apposita pagina e qualora si tenti di accedere a questa sezione senza essere autenticati si verrà reindirizzati alla home page del sito. Sarà anche presente in ogni pagina un menu laterale per navigare tra le varie pagine di amministrazione.
\subparagraph{Dashboard}
Questa sarà la pagina a cui si verrà reindirizzati una volta effettuata la log in, qui ci verrà fornito un messaggio che ci avvertirà con che account siamo loggati e sotto di esso sarà presente una \emph{tabella di log} che avviserà l' amministratore delle ultime azioni effettuate dagli amministratori del sito quali ad esempio l' inserimento di un nuovo evento, l' eliminazione o l' aggiunta di foto segnalando la data e l' ora di tali modifiche.
\subparagraph{Modifica menu}
In questa sezione si potrà modificare il menu attualmente in uso, inizialmente la pagina conterrà tutti i piatti del menu attualmente in uso, quindi ogni input text conterrà come value il valore del piatto attualmente presente nel menu, quindi l' amministratore che vorrà cambiare il menu potrà aggiungere o rimuovere piatti oppure modificare i piatti attuali modificandone solo il prezzo o il nome o entrambi.\\
I controlli che sono stati inseriti in questa pagina sono:
\begin{itemize}
	\item l' amministratore dovrà aver riempito tutti i campi prima di poter aggiornare il menu, qualora tenti di aggiornare il menu lasciando campi vuoti, l' operazione verrà interrotta e verrà notificato dei campi vuoti che verranno evidenziati da un' aura rossa intorno alle input text lasciate vuote. Tale aura verrà rimossa qualora venga inserito all' interno dell' input text un carattere valido;
	\item se si cerca di inserire qualcosa di diverso da un numero sulle input text che richiedono un prezzo, tale operazione verrà impedita fornendo un messaggio di errore;
\end{itemize}
\subparagraph{Modifica immagine}
In questa sezione verrà data la possibilità all' amministratore di modificare i dati di un' immagine quali nome, testo alternativo e titolo. Sono stati implementati dei controlli per non permettere il submit del form qualora ci sia uno o più campi lasciati vuoti.
\subparagraph{Inserimento nuovo evento}
Qui l' amministratore potrà creare un nuovo evento fornendone il titolo, la descrizione, il costo del menu, la data a cui si riferisce l' evento e le portate che lo compongono. Tutti questi elementi sono obbligatori, e come tali qualora si cerchi di eseguire l' input di tale form, i campi vuoti verranno evidenziati all' utente come nelle altre pagine.
\subparagraph{Inserimento nuova news}
Qui i campi saranno tutti necessari, quindi l' amministratore dovrà inserire tutti i campi, e qualora cerchi di eseguire il submit in presenza di campi vuoti non verrà permessa tale azione e verranno evidenziati i campi come nelle altre pagine della sezione privata.