\section{XML e XMLSchema}
Si è scelto di adottare XMLSchema al posto di DTD come linguaggio per descrivere la struttura dei file XML perchè più espressivo, oltre che a scopo propedeutico, essendo più complesso. Di seguito verranno spiegate alcune scelte riguardanti la struttura.

\paragraph{Scarso utilizzo degli attributi}

I motivi per cui gli attributi sono poco presenti sono tre:
\begin{itemize}
\item Un elemento è maggiormente estendibile rispetto ad un attributo;
\item Un attributo viene utilizzato per descrivere metadati, mentre i dati in se sono inseriti negli elementi;
\item Verbosità del linguaggio XMLSchema nel descrivere gli attributi.
\end{itemize}

\paragraph{Formato della data}
In molti file è richiesta una data. Si è scelto di salvare tale data in modo esteso, suddividendo l'elemento data in sottoelementi, cosi facendo è possibile indicare dei range di valore accettabili per giorno, mese ed anno. 
Non è stato utilizzato il tipo \emph{Data} offerto da XMLSchema per evitare di dover utilizzare il formato americano.

\paragraph{Elementi opzionali}
Nel file riguardante il menù sono presenti elementi non obbligatori. Nel caso di elementi semplici si è scelto di utilizzare l'attributo nillable="true" per indicare che tale elemento è opzionale. Nel file XML dunque l'elemento dovrà sempre essere presente, e nel caso in cui non lo si voglia definire vi si assegnerà xsi:nil="true".  
Nel caso di elementi complessi si è scelto di utilizzare l'attributo minOccurs="0" che permette di non inerire l'elemento nel file XML. Tale scelta è stata fatta per rendere più leggibili i file XML, che risulterebbero altrimenti appesantiti.

Questi approcci sono necessari in quanto lasciare semplicemente vuoto l'elemento comporterà un errore di validazione, tranne che per gli elementi di tipo stringa.

\paragraph{Modello adottato}
Il modello adottato è \emph{Bambole russe} in quanto non si è ritenuto il riutilizzo del codice di primaria importanza.