\section{Analisi dell'utenza}

Una delle prime operazioni che sono state svolte nella realizzazione del progetto è stata un'attenta analisi dell'\textbf{utenza}. Da ciò ne sono derivate diverse considerazioni, che hanno fatto sì che il progetto si sviluppasse in una direzione rispetto ad un altra. Naturalmente quando si parla di utenza il \textit{focus} viene messo sull'interfaccia del sito, che comprende, oltre al modo in cui esso appare all'utente, anche al modo in l'utente interagisce con esso. \\ \\
Essendo nel mondo della ristorazione possiamo assumere che la quasi totalità degli utenti che effettuano una visita e una navigazione all'interno del sito sono dei \textbf{possibili clienti}, ovvero utenti che hanno come scopo quello di prendere in considerazione l'idea di cenare o pranzare nel ristorante. Ecco perché, come già evidenziato nel capitolo precedente, si è scelto di rendere il sito come una \textbf{vetrina}. \\ \\
L'utente deve poter entrare nel sito e vivere una esperienza il più simile possibile a quella che avrebbe nell'entrare nel ristorante. Naturalmente stiamo parlando di due esperienze diverse, la prima delle quali non comprende sensazioni come l'udito e l'olfatto, anch'essi determinanti, ma dal punto di vista visivo la sensazione dovrebbe essere analoga, almeno per chi ha la fortuna di percepire questo senso. Ecco perché lungo tutta la navigazione è stato scelto di far scorrere una \textit{slideshow} di immagini raffiguranti delle sezioni topiche, ovvero:

\begin{itemize}

	\item Un'immagine dell'esterno del ristorante;
	\item Un'immagine dell'interno del ristorante;
	\item Un'immagine della cantina;
	\item Un'immagine di un piatto;

\end{itemize}
Allo stato attuale del progetto le immagini fornite risultano poco definite e quindi vengono visualizzate sgranate. L'effetto è chiaramente poco gradevole e per quanto riguarda l'accessibilità è chiaramente un danno. In futuro, non appena si otterranno immagini migliori il problema verrà indubbiamente risolto. Allo stato attuale è stato scelto di mantenerle per raggiungere l'obiettivo spiegato.\\
Per quanto riguarda gli utenti svantaggiati sono stati effettuati gli opportuni test di accessisbilità ed è stato verificato che il sito sia accessibile a utenti ipovedenti.\\ \\
La parte di amministrazione del sito ha un utenza limitata a pochi elementi, che potranno presumibilmente essere il gestore del ristorante e/o altri componenti del personale. Per questa sezione è stato mantenuto lo stesso stile, ma per ovvi motivi non si è puntato fortemente il dito sull'estetica così come è stato fatto per la restante parte.
