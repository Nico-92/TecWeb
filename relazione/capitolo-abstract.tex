\section{Abstract}

I ristoranti al giorno d'oggi, soprattutto in Italia, nonostante la crisi economica martellante restano ancora una meta molto gradita e quasi irrinunciabile per la maggior parte delle persone. La ristorazione è un settore molto interessante per chiunque ami la buona cucina e la buona compagnia. Ma qualsiasi ristoratore sa benissimo che non basta una cucina raffinata per avere successo. Gran parte dell'appetibilità di un esercizio di questo tipo dipende moltissimo da \textbf{come esso appare} e da \textbf{come la gente ne parla}. In mezzo a questi due pilastri si colloca perfettamente il web. Gli utenti navigano ininterrottamente nella rete e oggigiorno essa è diventata la fonte primaria di informazioni. Risulta dunque fondamentale per qualsiasi ristoratore dare una buona immagine di sé e del proprio locale nel web. Viviamo nell'era dei \textit{social network} quindi siamo perfettamente consapevoli del fatto che gran parte della pubblicità che un esercizio può ottenere lo attinge da essi. Inutile citare \textit{facebook} o \textit{twitter}, ormai padroni assoluti. Più interessante in questo settore sarebbe parlare di \textit{tripadvisor} ma lo scopo di questo progetto non è certo quello di analizzare questi aspetti. La domanda che viene spontanea, date queste considerazioni, è come un sito web si inserisce in questo insieme e quale sia la sua reale utilità.

Bisogna essere realisti: l'utilità di un sito web per un ristorante è \textbf{molto bassa}, o almeno questa è la situazione al giorno d'oggi. Se fino a qualche anno fa per un ristoratore l'avere o il non avere un sito web online poteva fare un'enorme differenza nel 2014 questa differenza si è molto assottigliata. Pensiamo per esempio ai contenuti significativi che il sito web da noi sviluppato offre:

\begin{itemize}

	\item Possibilità di visualizzare una \textbf{galleria} di immagini;
	\item Possibilità di visualizzare \textbf{news};
	\item Possibilità di visualizzare il \textbf{menu};
	\item Possibilità di visualizzare gli \textbf{eventi}.

\end{itemize}

Chiunque abbia un minimo di familiarità con \textit{facebook} sa benissimo che tutto ciò può essere facilmente gestito tramite esso. Anzi, considerando che un utente medio appena apre il proprio browser una delle prime pagine che visita è proprio \textit{facebook} risulta chiaro come l'inglobare il tutto su di esso porti a una visibilità notevolmente maggiore. Nessun utente aprirebbe mai \texttt{levecchiecredenze.it} di routine, semmai lo farà una prima volta e in seguito solamente quando lo riterrà necessario.

Detto questo sicuramente un pensiero che potrebbe scorrere nella testa di chiunque è questa:

\begin{center}

\textit{Perchè mai dovrei sviluppare un sito web di un ristorante?}

\end{center}
La risposta, viste le precedenti considerazioni, non sembra così immediata. Naturalmente risulta chiaro come nell'interesse primario di un ristoratore ci sia l'afflusso di persone nel suo ristorante. Anzi, probabilmente questa è l'unica finalità. Se la domanda ci venisse posta in questo momento queste sarebbero le risposte che daremmo:

\begin{itemize}

	\item Anzitutto si tratta di un progetto didattico e la realizzazione delle sue componenti rispecchia in modo coerente gli obiettivi e gli argomenti presentati dal corso;
	\item In secondo luogo riteniamo che i social network in questo contesto abbiano un grosso difetto: forniscono un'\textbf{interfaccia comune} e non personalizzabile. Vedere la pagina \textit{facebook} di un ristorante piuttosto che di un altro non farebbe grossa differenza se non per il contenuto. Riteniamo dunque che la componente \textbf{presentazionale} sia fondamentale per il sito web di un ristorante;
	\item Riteniamo che, se un ristorante possiede un sito web, allora i social network possono fungere solamente da \textbf{\textit{passerella}} tra l'utente ed esso. Se il sito web risulta gradevole alla vista, accattivante e presenta i contenuti in maniera ordinata allora esso può risultare molto più efficace rispetto ad un social network ed invoglierebbe maggiormente l'utente a frequentare il locale;
	\item Per ultimo, ma non meno importante, riteniamo che l'\textbf{accessibilità} sia un concetto fondamentale nel web, e molto spesso i social network non rispettano questo canone. Il sito web è stato realizzato prestando attenzione massima all'accessibilità, favorendo tutte le categorie di utenti possibili nel migliore dei modi e il tutto senza compromettere il suo stile elegante. Riteniamo che le categorie di utenti svantaggiati possano trarre un'esperienza migliore visitando il sito piuttosto che visualizzando la sua pagina in un social network.
	
\end{itemize}

"\textit{Le vecchie credenze}" risiede appieno in questa realtà. È in primo luogo un ristorante "\textit{di lusso}", con prezzi molto alti e le cui aspettative sono di conseguenza elevate. Proprio per creare grosse aspettative si è scelto di sviluppare questo sito, in modo da presentare il ristorante esattamente com'è: elegante, lussuoso e dalla cucina trascendentale. Con questo non vogliamo certo fare della pubblicità nel presente documento, si tratta solamente di giustificare al meglio le scelte e lo stile adottato. Se avessimo dovuto sviluppare un sito per una pizzeria avremmo certamente utilizzato uno stile differente. Ma, dal momento che si tratta di un ristorante elitario, abbiamo concentrato molto l'attenzione nel presentare al meglio ciò che esso offre, in modo che chiunque lo frequenti sappia cosa esso offre e si crei un'aspettativa alta.

Questo è in sostanza "\textit{levecchiecredenze.it}", un sito di un ristorante che si pone come obiettivo quello di presentare i contenuti in maniera accessibile e strutturata e allo stesso tempo fornire un'interfaccia elegante e pomposa, allo scopo di invogliare chiunque effettui una visita a pranzare o cenare presso di esso.